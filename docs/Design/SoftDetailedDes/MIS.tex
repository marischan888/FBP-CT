\documentclass[12pt, titlepage]{article}

\usepackage{amsmath, mathtools}

\usepackage[round]{natbib}
\usepackage{amsfonts}
\usepackage{amssymb}
\usepackage{graphicx}
\usepackage{colortbl}
\usepackage{xr}
\usepackage{hyperref}
\usepackage{longtable}
\usepackage{xfrac}
\usepackage{tabularx}
\usepackage{float}
\usepackage{siunitx}
\usepackage{booktabs}
\usepackage{multirow}
\usepackage[section]{placeins}
\usepackage{caption}
\usepackage{fullpage}
\usepackage{amsmath}
\usepackage{amssymb}

\hypersetup{
bookmarks=true,     % show bookmarks bar?
colorlinks=true,       % false: boxed links; true: colored links
linkcolor=red,          % color of internal links (change box color with linkbordercolor)
citecolor=blue,      % color of links to bibliography
filecolor=magenta,  % color of file links
urlcolor=cyan          % color of external links
}

\usepackage{array}

\externaldocument{../../SRS/SRS}

\input{../../Comments}
\input{../../Common}

\usepackage{gensymb}
\usepackage{soul,xcolor}
\newcommand{\add}{\textcolor{red}}
\usepackage{lipsum}
\usepackage[many]{tcolorbox}

\newtcolorbox{cross}{blank,breakable,parbox=false,
  overlay={\draw[red,line width=5pt] (interior.south west)--(interior.north east);
    \draw[red,line width=5pt] (interior.north west)--(interior.south east);}}

\begin{document}
\setstcolor{red}
\title{Module Interface Specification for \progname{}}

\author{\authname}

\date{\today}

\maketitle

\pagenumbering{roman}

\section{Revision History}

\begin{tabularx}{\textwidth}{p{3cm}p{2cm}X}
\toprule {\bf Date} & {\bf Version} & {\bf Notes}\\
\midrule
March 11 & 1.0 & Initial document\\
\bottomrule
\end{tabularx}

~\newpage

\section{Symbols, Abbreviations and Acronyms}

See SRS Documentation at \href{https://github.com/marischan888/FBP-CT/blob/main/docs/SRS/SRS.pdf}{SRS}.

\noindent \begin{tabular}{l l l}
\toprule
\textbf{symbol} & \textbf{unit} & \textbf{description}\\
\midrule
$\mathcal{F}$ & None & Fourier Transform \\
$\mathcal{F}^{-1}$ & None & Inverse Fourier Transform \\
$d$ & None & diagonal length of padded image \\
\bottomrule
\end{tabular}\\

\newpage

\tableofcontents

\newpage

\pagenumbering{arabic}

\section{Introduction}

The following document details the Module Interface Specifications for \progname.
Complementary documents include the System Requirement Specifications
and Module Guide.  The full documentation and implementation can be
found at \href{https://github.com/marischan888/FBP-CT/tree/main/docs/Design/SoftDetailedDes}{MIS}.

\section{Notation}
The structure of the MIS for modules comes from \citet{HoffmanAndStrooper1995},
with the addition that template modules have been adapted from
\cite{GhezziEtAl2003}.  The mathematical notation comes from Chapter 3 of
\citet{HoffmanAndStrooper1995}.  For instance, the symbol := is used for a
multiple assignment statement and conditional rules follow the form $(c_1
\Rightarrow r_1 | c_2 \Rightarrow r_2 | ... | c_n \Rightarrow r_n )$.

The following table summarizes the primitive data types used by \progname.

\begin{center}
\renewcommand{\arraystretch}{1.2}
\noindent
\begin{tabular}{l l p{7.5cm}}
\toprule
\textbf{Data Type} & \textbf{Notation} & \textbf{Description}\\
\midrule
character & char & a single symbol or digit\\
String & None & a char array\\
integer & $\mathbb{Z}$ & a number without a fractional component in (-$\infty$, $\infty$) \\
real & $\mathbb{R}$ & any number in (-$\infty$, $\infty$)\\
1D real number array with size s & $\mathbb{R}^(s)$ & any number in (-$\infty$, $\infty$)\\
2D real number array with size m & $\mathbb{R}^{m \times m}$ & any number in (-$\infty$, $\infty$)\\
\bottomrule
\end{tabular}
\end{center}

\noindent
The specification of \progname \ uses some derived data types: sequences, strings, and
tuples. Sequences are lists filled with elements of the same data type. Strings
are sequences of characters. Tuples contain a list of values, potentially of
different types. In addition, \progname \ uses functions, which
are defined by the data types of their inputs and outputs. Local functions are
described by giving their type signature followed by their specification.

\section{Module Decomposition}

The following table is taken directly from the Module Guide document for this project.

\begin{table}[h!]
\centering
\begin{tabular}{p{0.3\textwidth} p{0.6\textwidth}}
\toprule
\textbf{Level 1} & \textbf{Level 2}\\
\midrule

{Hardware-Hiding} & Hardware-Hiding Module \\
\midrule

{Behaviour-Hiding}
& Filter Module\\
& FBP Module\\
& \add{Sinogram Simulation Module}\st{IO Handler Module}\\
\midrule

{Software Decision} & Services Module\\
& \add{IO Handler Module}\st{Sinogram Simulation Module}\\
\bottomrule

\end{tabular}
\caption{Module Hierarchy}
\label{TblMH}
\end{table}

\newpage
~\newpage

\section{MIS of Filter Module} \label{ModuleF}

\subsection{Module}
Filter

\subsection{Uses}

\subsection{Syntax}

\subsubsection{Exported Constants}
None

\subsubsection{Exported Access Programs}

\begin{center}
\begin{tabular}{|l|l|l|l|}
\hline
\textbf{Name} & \textbf{In} & \textbf{Out} & \textbf{Exceptions} \\
\hline
get\_fourier\_filter & s: \(\mathbb{Z}\), filter\_name: \add{Enum}\st{String} & \st{filter: }\(\mathbb{R}^{s}\)  & ValueError, NULL \\
\hline
\end{tabular}
\end{center}

\subsection{Semantics}

\subsubsection{State Variables}
None

\subsubsection{Environment Variables}
None

\subsubsection{Assumptions}
\add{None}\st{Input size s is a positive number.}

\subsubsection{Access Routine Semantics}

\noindent get\_filter(s, filter\_name):
\begin{itemize}
\item output: \((\texttt{filter\_name} = \texttt{SHEPP} \Rightarrow f(s) \mid\mid
  \texttt{filter\_name} = \texttt{RAMP} \Rightarrow g(s) \mid\mid \texttt{filter\_name} =
  \texttt{NONE} \Rightarrow None)\) where:\\
  \begin{enumerate}
    \item
      \begin{equation}
        f(s) =
        \left\{
          \begin{aligned}
            0.25 &, & s = 0 \\
            \frac{-1}{{\pi s}^2} &, & \text{s is odd}\\
            0 &, & \text{s is even}\\
          \end{aligned}
        \right.
      \end{equation}
    \item
      \begin{equation}
        g(s) =
        \frac{\sin{\pi\cdot freq(n)}}{\pi\cdot freq(n)} \cdot freq(n)
      \end{equation}
      freq(n) corresponds to the frequency bins in the discrete Fourier domain:\\
      \begin{equation}
        freq(n) =
        \left\{
          \begin{aligned}
            \frac{n}{s} &, & 0 \leq n < \frac{s}{2} \\
            \frac{n - s}{s} &, & \frac{s}{2} \leq n < s \\
          \end{aligned}
        \right.
      \end{equation}
    \end{enumerate}
  \item exception: \(exc := (s < 0 \Rightarrow ValueError \mid filter\_name \notin \{\texttt{SHEPP, RAMP, NONE}\} \Rightarrow NULL)\)
\end{itemize}


\subsubsection{Local Functions}
None

\newpage

\section{MIS of FBP Module} \label{ModuleFBP}

\subsection{Module}
FBP

\subsection{Uses}
Filter Module

\subsection{Syntax}

\subsubsection{Exported Constants}
None

\subsubsection{Exported Access Programs}

\begin{center}
  \begin{tabular}{|l|m{18em}|l|m{11em}|}
\hline
\textbf{Name} & \textbf{In} & \textbf{Out} & \textbf{Exceptions} \\
\hline
reconstruct & sinogram: \(\mathbb{R}^{m \times m}\), \(\add{\theta}: \mathbb{R}^{n}\), filter\_name: \add{enum}, os: int & image: \(\mathbb{R}^{os \times os}\)& ValueError, EmptyArrayError\\
\hline
\end{tabular}
\end{center}

\subsection{Semantics}

\subsubsection{State Variables}
\st{2D array with size s as the fourier filter.} \add{None}

\subsubsection{Environment Variables}
None

\subsubsection{Assumptions}
\add{None}\st{Input sinogram are valid 2D frequency spectra.}

\subsubsection{Access Routine Semantics}

\noindent reconstruct(sinogram, theta, filter\_name, os):
\begin{itemize}
\item output: \(\mathcal{F}^{-1}(\mathcal{F}(sinogram))\) where:
  \begin{description}
      \item[Fourier Transform]
    \[ F(\omega) = \int_{-\infty}^{\infty} f(t) \, e^{-i \omega t} \, dt\]
    \item[Inverse Fourier Transform]
      \[ f(t) = \frac{1}{2\pi} \int_{-\infty}^{\infty} F(\omega) \, e^{i \omega t} \, d\omega\]
     \add{ \(\omega\): rad/s and t is the (x,y) position in spatial domain}
  \end{description}

  \item exception: exc :=
    \begin{center}
      \begin{tabular}{|m{15em}|m{15em}|}
        \hline
        \textbf{Exceptions} & \textbf{Description} \\
        \hline
        \(os < 0 \mid\mid theta < 0 \Rightarrow ValueError\) & Valid output size and interpolation angle should not be negative. \\
        \hline
        \(sinogram.size < 0 \Rightarrow EmptyArrayError\) & The input sinogram is invalid if the input array size is smaller than 0.\\
        \hline
      \end{tabular}
    \end{center}
\end{itemize}


\subsubsection{Local Functions}
None

\newpage
\section{MIS of Sinogram Simulation Module} \label{ModuleS}

\subsection{Module}
Sinogram Simulation

\subsection{Uses}
None

\subsection{Syntax}

\subsubsection{Exported Constants}
None

\subsubsection{Exported Access Programs}

\begin{center}
\begin{tabular}{|l|m{18em}|l|l|}
\hline
\textbf{Name} & \textbf{In} & \textbf{Out} & \textbf{Exceptions} \\
\hline
radon & image: \(\mathbb{R}^{m \times m}\)\st{, preserve\_angle: Boolean}, \(\theta\): \(\mathbb{R}^{n}\) & sinogram: \(\mathbb{R}^{m \times m}\) & None \\
\hline
\end{tabular}
\end{center}

\subsection{Semantics}

\subsubsection{State Variables}
None

\subsubsection{Environment Variables}
None

\subsubsection{Assumptions}
Input Image are valid 2D frequency spectra.

\subsubsection{Access Routine Semantics}

\noindent
radon(image\st{, preserve\_angle}, theta):
\begin{itemize}
\item output:= \(image \Rightarrow \int_{-\infty}^{\infty} imgae(x\cos(\theta) + y\sin(\theta),
        x\sin(\theta) - y\cos(\theta))dy\)
\item exception: None
\end{itemize}

\subsubsection{Local Functions}
\add{None}
\begin{cross}
\noindent
padding\_radon(image, preserve\_angle):
\begin{itemize}
\item output: padded\_image and center.
  \begin{description}
    \item[padded\_image] :=
      \begin{itemize}
      \item The diagonal length of the padded square image is calculated as:
        \(d = \sqrt{2}\cdot max(image.size)\)
      \item The required padding size for each dimension is: \(|d - s|, \forall s \in image.size\)
      \end{itemize}
    \item[center] := \(\frac{padded\_image[0]}{2}\)
  \end{description}
\item exception: If image is not an 2D array, raise ValueError
\end{itemize}
\end{cross}

\newpage
\section{MIS of IO Handler Module} \label{ModuleIO}

\subsection{Module}
IO Handler

\subsection{Uses}
\st{
  FBP Module\\
  Sinogram Simulation Module\\
  Services Module\\
  Hardware Handling Module\\
}

\subsection{Syntax}

\subsubsection{Exported Constants}
None

\subsubsection{Exported Access Programs}
\begin{cross}
\begin{center}
  \begin{tabular}{|l|m{10em}|l|m{8em}|}
  \hline
  \textbf{Name} & \textbf{In} & \textbf{Out} & \textbf{Exceptions} \\
  \hline
    process\_service & file: \(\mathbb{R}^{m \times m}\), theta: \(\mathbb{R}^{m}\), filter\_name: String, os: int, preserve\_angle: Boolean & None & None \\
  \hline
  get\_reconstruction & None & reconstruction: \(\mathbb{R}^{os \times os}\) & None \\
  \hline
\end{tabular}
\end{center}
\end{cross}
\add{None}
\subsection{Semantics}

\subsubsection{State Variables}
\st{
  service: service type\\
  sinogram: a 2D array with size m\\
  reconstructed\_image: a 2D array with size os\\
}

\subsubsection{Environment Variables}
None

\subsubsection{Assumptions}
\add{None}\st{Input Image are valid 2D frequency spectra.}

\subsubsection{Access Routine Semantics}
\begin{cross}
\noindent
select\_services(op):
\begin{itemize}
\item transition:  service := \(op ='reconstruction' \Rightarrow  0 \mid\mid op = 'verification' \Rightarrow  1\)
\item exception: exc := \(op \notin ['reconstruction', 'verification'] \Rightarrow ValueError\)
\end{itemize}

\noindent
process\_service(file, theta, filter\_name, os, preserve\_angle):
\begin{itemize}
\item transition:
  \begin{description}
  \item[sinogram :=] \(service = 0 \Rightarrow load\_image(file, preserve\_angle) \mid\mid
    service = 1 \Rightarrow simulation.radon()\)
  \item[reconstructed\_image :=] fbp.reconstruct() from Module \ref{ModuleFBP}.
  \end{description}
\item exception: None
\end{itemize}

\noindent
get\_reconstruction(None):
\begin{itemize}
\item output: out := reconstructed\_image
\item exception: None
\end{itemize}
\end{cross}

\subsubsection{Local Functions}
\noindent
load\_image(file\st{, preserve\_angle}):
\begin{itemize}
\item output: read user-input image file and return a 2D numpy array.
\item exception: exc := file does not exit then FileNotFound.
\end{itemize}
\noindent
\add{
save\_image(image):
\begin{itemize}
\item output: read reconstructed image (2D numpy array) from fbp and save jpg image
  file to selected local path.
\item exception: exc := path does not exit then PathNotFound.
\end{itemize}
}
\newpage
\section{MIS of Services Module} \label{ModuleCon}

\subsection{Module}
Services

\subsection{Uses}
\add{FBP Module, Simulation Module, IO Handler Module}

\subsection{Syntax}

\subsubsection{Exported Constants}
\add{None}\st{
  VERIFICATION = 0\\
  RECONSTRUCTION = 1\\
}

\subsubsection{Exported Access Programs}
None

\subsection{Semantics}

\subsubsection{State Variables}
None

\subsubsection{Environment Variables}
None

\subsubsection{Assumptions}
None

\subsubsection{Access Routine Semantics}
None

\subsubsection{Local Functions}
\noindent
\add{
reconstruct(sinogram: \(\mathbb{R}^{m \times m}\), \(\theta: \mathbb{R}^{n}\), filter\_name: enum, os: int):
\begin{itemize}
\item it is just a API Gateway for user calling the reconstruction service in
  Module\ref{ModuleFBP} and provide saving result option using Module\ref{ModuleIO}.
\item output: None.
\item exception: None
\end{itemize}
}
\noindent
\add{
simulate(image: \(\mathbb{R}^{m \times m}\), \(\theta: \mathbb{R}^{n}\), filter\_name: enum, os: int):
\begin{itemize}
\item it is just a API Gateway for user calling the simulation service in
  Module\ref{ModuleS} and provide saving result option using Module\ref{ModuleIO}.
\item output: None.
\item exception: None
\end{itemize}
}

\newpage

\bibliographystyle {plainnat}
\bibliography {../../../refs/References}

\newpage

\section{Appendix} \label{Appendix}

\wss{Extra information if required}


\end{document}
