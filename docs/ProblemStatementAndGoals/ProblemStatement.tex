\documentclass{article}

\usepackage{tabularx}
\usepackage{booktabs}

\title{Problem Statement and Goals\\\progname}

\author{\authname}

\date{\today}

\input{../Comments}
%% Common Parts

\newcommand{\progname}{FBP CT Image Reconstruction} % PUT YOUR PROGRAM NAME HERE
\newcommand{\authname}{Qianlin Chen} % AUTHOR NAMES

\usepackage{hyperref}
    \hypersetup{colorlinks=true, linkcolor=blue, citecolor=blue, filecolor=blue,
                urlcolor=blue, unicode=false}
    \urlstyle{same}
                                


\begin{document}

\maketitle

\begin{table}[hp]
\caption{Revision History} \label{TblRevisionHistory}
\begin{tabularx}{\textwidth}{llX}
\toprule
\textbf{Date} & \textbf{Developer(s)} & \textbf{Change}\\
\midrule
  January 16, 2025 & Qianlin Chen & Initial Draft \\
\midrule
  ... & ... & ...\\
\bottomrule
\end{tabularx}
\end{table}

\section{Problem Statement}

\subsection{Problem}
% Background
With the arrival of Computed Tomography (CT) as a diagnostic tool in medical
imaging, X-ray imaging underwent a revolution. Tomography is a method of imaging
a two- or three-dimensional object from multiple one-dimensional ``slices'' of
the object. In a CT scan, these slices are created by multiple parallel X-ray
beams passing through the object at varying angles. The initial and final
intensity of each beam is recorded, and the original image is reconstructed
using backprojection with data from multiple slices.
%% What is the problem and Why is this problem matter:
%% TODO: citation of problem
\newline However, significant noise blurs the recreated image, even as the number of
backprojections increases. Regardless of the number of directions used for
backprojection, it cannot perfectly recreate the image using the backprojection
formula. Therefore, it is necessary to develop techniques to filter out noise
created by backprojection and produce a smoother representation of the object.
Additionally, different filtering techniques may yield varying reconstruction
efficiencies, so selecting an appropriate filter is crucial.

\subsection{Inputs and Outputs}
\subsubsection{Inputs}
\begin{itemize}
\item Phantom images.
\item Sinogram data.
\item Projection angles in degrees.
\item Filter tupe to be applied during backprojection.
\end{itemize}

\subsubsection{Outputs}
\begin{itemize}
  \item Reconstructed CT images.
\end{itemize}

\subsection{Stakeholders}
\begin{itemize}
  \item Researchers/Company that want to use the CT image reconstruction with
    filter techniques.
\end{itemize}

\subsection{Environment}
\begin{description}
\item[Software] Windows, Linux or Mac OS
 %% do I need hardware like CT scanner? \item[Hardware] 
\end{description}

\section{Goals}
\begin{description}
\item[CT images qualities improvement]\newline The product should provide at least three
  types of visual effects to show the instructor's motions in Tai Chi lectures,
  making it easier to see the instructor's motions in video calls.
\item[Images reconstruction efficient improvement] \newline
\item[Ease of use] \newline The application should be intuitive to use and no further
  instruction is required. People of all age groups should be able to learn all
  the features of the application with ease.
\end{description}

\section{Stretch Goals}

\section{Challenge Level and Extras}

\wss{State your expected challenge level (advanced, general or basic).  The
challenge can come through the required domain knowledge, the implementation or
something else.  Usually the greater the novelty of a project the greater its
challenge level.  You should include your rationale for the selected level.
Approval of the level will be part of the discussion with the instructor for
approving the project.  The challenge level, with the approval (or request) of
the instructor, can be modified over the course of the term.}

\wss{Teams may wish to include extras as either potential bonus grades, or to
make up for a less advanced challenge level.  Potential extras include usability
testing, code walkthroughs, user documentation, formal proof, GenderMag
personas, Design Thinking, etc.  Normally the maximum number of extras will be
two.  Approval of the extras will be part of the discussion with the instructor
for approving the project.  The extras, with the approval (or request) of the
instructor, can be modified over the course of the term.}

\newpage{}

\section*{Appendix --- Reflection}

\wss{Not required for CAS 741}

\input{../Reflection.tex}

\begin{enumerate}
    \item What went well while writing this deliverable?
    \item What pain points did you experience during this deliverable, and how
    did you resolve them?
    \item How did you and your team adjust the scope of your goals to ensure
    they are suitable for a Capstone project (not overly ambitious but also of
    appropriate complexity for a senior design project)?
\end{enumerate}

\end{document}
