\documentclass{article}

\usepackage{tabularx}
\usepackage{booktabs}
\usepackage{biblatex}
\addbibresource{../../refs/References.bib}

\title{Problem Statement and Goals\\\progname}

\author{\authname}

\date{\today}

\input{../Comments}
%% Common Parts

\newcommand{\progname}{FBP CT Image Reconstruction} % PUT YOUR PROGRAM NAME HERE
\newcommand{\authname}{Qianlin Chen} % AUTHOR NAMES

\usepackage{hyperref}
    \hypersetup{colorlinks=true, linkcolor=blue, citecolor=blue, filecolor=blue,
                urlcolor=blue, unicode=false}
    \urlstyle{same}
                                


\begin{document}

\maketitle

\begin{table}[hp]
\caption{Revision History} \label{TblRevisionHistory}
\begin{tabularx}{\textwidth}{llX}
\toprule
\textbf{Date} & \textbf{Developer(s)} & \textbf{Change}\\
\midrule
  January 16, 2025 & Qianlin Chen & Initial Draft \\
\midrule
  ... & ... & ...\\
\bottomrule
\end{tabularx}
\end{table}

\section{Problem Statement}

\subsection{Problem}
% Background
With the arrival of Computed Tomography (CT) as a diagnostic tool in medical
imaging, X-ray imaging underwent a revolution. Tomography is a method of imaging
a two- or three-dimensional object from multiple one-dimensional ``slices'' of
the object. In a CT scan, these slices are created by multiple parallel X-ray
beams passing through the object at varying angles. The initial and final
intensity of each beam is recorded, and the original image is reconstructed
using backprojection with this data from multiple slices.
%% What is the problem and Why is this problem matter:
\newline However, significant noise blurs the recreated image, even as the number of
backprojections increases. Regardless of the number of directions used for
backprojection, it can not perfectly recreate the image using the backprojection
formula\cite{Beatty2012}. Therefore, it is necessary to develop techniques to filter out noise
created by backprojection and produce a smoother representation of the object.
Additionally, different filtering techniques may yield varying reconstruction
efficiencies, so selecting an appropriate filter is crucial.

\subsection{Inputs and Outputs}
\subsubsection{Inputs}
\begin{itemize}
\item Phantom images.
\item Sinogram data.
\item Projection angles.
\item Filter type to be applied during backprojection.
\end{itemize}

\subsubsection{Outputs}
\begin{itemize}
  \item Reconstructed CT images.
\end{itemize}

\subsection{Stakeholders}
\begin{itemize}
\item Medical Researchers who require advanced imaging tools for studying
  medical conditions and treatment outcomes.
\item Hospital which seeks efficient and accurate imaging technologies.
\end{itemize}

\subsection{Environment}
\begin{description}
\item[Software] \hfill \\ Windows, Linux or Mac OS
\end{description}

\section{Goals}
\begin{description}
\item[High-quality CT Image Reconstruction] \hfill \\ The tool should improve
  the quality of reconstructed CT images through advanced filtering techniques.
\item[Filter Options] \hfill \\ The tool should provide high-pass and
  low-pass filter options to filter different types of noise.
\item[User-Friendly Design] \hfill \newline The tool will be intuitive, requiring
  no additional instructions for users to understand all features.
\end{description}

\section{Stretch Goals}
\begin{description}
\item[Adaptive Filtering] \hfill \\ The tool should implement adaptive
  filters that automatically adjust based on image characteristics.
\item[Real-Time Reconstruction] \hfill \\ The tool should support real-time
  image reconstruction during the scanning process, enabling faster diagnostics.
\end{description}

\section{Challenge Level and Extras}
The main challenge lies in the integration of domain knowledge from the medical
and mathematical fields. Understanding the principles of medical imaging,
including tomography and the Radon transform, as well as mastering the
associated mathematical concepts, requires additional effort.\\
Testing poses another difficulty in this project. For functional testing, the
lack of access to high-quality datasets limits the ability to evaluate the
accuracy and effectiveness of the reconstruction process. For usability testing,
reaching potential users, such as researchers in medical imaging or related
fields, is difficult due to their limited availability.\\
The expected challenge level for this project is general. While it requires
medical and mathematical knowledge, many research resources are available
online. The CT image reconstruction process is well-documented and feasible for
implementation. This project is appropriate for graduate-level study, combining
computer science with interdisciplinary applications.\\

\newpage
\printbibliography

\end{document}
