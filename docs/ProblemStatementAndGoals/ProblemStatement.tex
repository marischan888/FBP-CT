\documentclass{article}

\usepackage{tabularx}
\usepackage{booktabs}
\usepackage{biblatex}
\addbibresource{../../refs/References.bib}

\title{Problem Statement and Goals\\\progname}

\author{\authname}

\date{\today}

\input{../Comments}
%% Common Parts

\newcommand{\progname}{FBP CT Image Reconstruction} % PUT YOUR PROGRAM NAME HERE
\newcommand{\authname}{Qianlin Chen} % AUTHOR NAMES

\usepackage{hyperref}
    \hypersetup{colorlinks=true, linkcolor=blue, citecolor=blue, filecolor=blue,
                urlcolor=blue, unicode=false}
    \urlstyle{same}
                                

\usepackage{gensymb}
\usepackage{soul,xcolor}
\newcommand{\add}{\textcolor{red}}
\usepackage{lipsum}
\usepackage[many]{tcolorbox}

\newtcolorbox{cross}{blank,breakable,parbox=false,
  overlay={\draw[red,line width=5pt] (interior.south west)--(interior.north east);
    \draw[red,line width=5pt] (interior.north west)--(interior.south east);}}

\begin{document}
\setstcolor{red}

\maketitle

\begin{table}[hp]
\caption{Revision History} \label{TblRevisionHistory}
\begin{tabularx}{\textwidth}{llX}
\toprule
\textbf{Date} & \textbf{Developer(s)} & \textbf{Change}\\
\midrule
  January 16, 2025 & Qianlin Chen & Initial Draft \\
  April 16, 2025 & Qianlin Chen & Final Document \\
\bottomrule
\end{tabularx}
\end{table}

\section{Problem Statement}
The following section highlights the challenges in reconstructing clear CT
images during the backprojection process, with a particular focus on selecting
appropriate filtering techniques. Addressing these challenges is crucial for
developing solutions that improve image clarity and enhance diagnostic accuracy.

\subsection{Problem}
% Background
With the arrival of Computed Tomography (CT) as a diagnostic tool in medical
imaging, X-ray imaging underwent a revolution. Tomography is a method of imaging
a two- or three-dimensional object from multiple one-dimensional ``slices'' of
the object. In a CT scan, these slices are created by multiple parallel X-ray
beams passing through the object at varying angles. The initial and final
intensity of each beam is recorded, and the original image is reconstructed
using backprojection with this data from multiple slices \cite{Beatty2012}.
%% What is the problem and Why is this problem matter:
\newline However, significant noise blurs the recreated image, even as the number of
backprojections increases. Regardless of the number of directions used for
backprojection, it can not perfectly recreate the image using the backprojection
formula\cite{Beatty2012}. Therefore, it is necessary to develop techniques to filter out noise
created by backprojection and produce a smoother representation of the object.
Additionally, different filtering techniques may yield varying reconstruction
efficiencies, so selecting an appropriate filter is crucial \cite{Lyra2011}.

\subsection{Inputs and Outputs}
\subsubsection{Inputs}
\begin{itemize}
\item \add{Image (for Sinogram data simulation).}\st{Phantom images}.
\item Sinogram data \add{(for reconstruction).}
\item Projection angles.
\item Filter type to be applied during backprojection.
\end{itemize}

\subsubsection{Outputs}
\begin{itemize}
  \item Reconstructed CT images.
  \item \add{Simulated Sinogram data}
\end{itemize}

\subsection{Stakeholders}
\begin{itemize}
\item Medical Researchers who require advanced imaging tools for studying
  medical conditions and treatment outcomes.
\item Hospital which seeks efficient and accurate imaging technologies.
\end{itemize}

\subsection{Environment}
\begin{description}
\item[Software] \hfill \\ Windows, Linux or Mac OS
\end{description}

\section{Goals}
\begin{description}
\item[High-quality CT Image Reconstruction] \hfill \\ The tool should improve
  the quality of reconstructed CT images through advanced filtering techniques.
\item[Filter Options] \hfill \\ The tool should provide high-pass and
  low-pass filter options to filter different types of noise.
 \item[Sinogram Simulation] \hfill \\ \add{This tool should provide a CT detector
     simulation to generate sinogrm given any CT image.}
\end{description}

\section{Stretch Goals}
\begin{description}
\item[Adaptive Filtering] \hfill \\ The tool should implement adaptive
  filters that automatically adjust based on image characteristics.
\item[Real-Time Reconstruction] \hfill \\ The tool should support real-time
  image reconstruction during the scanning process, enabling faster diagnostics.
\end{description}

\section{Challenge Level and Extras}
\st{
  The main challenge lies in the integration of domain knowledge from the medical
  and mathematical fields. Understanding the principles of medical imaging,
  including tomography and the Radon transform, as well as mastering the
  associated mathematical concepts, requires additional effort.

  Testing poses another difficulty in this project. For functional testing, the
  lack of access to high-quality datasets limits the ability to evaluate the
  accuracy and effectiveness of the reconstruction process. For usability testing,
  reaching potential users, such as researchers in medical imaging or related
  fields, is difficult due to their limited availability.
}

The expected challenge level for this project is general. While it requires
medical and mathematical knowledge, many research resources are available
online. The CT image reconstruction process is well-documented and feasible for
implementation. This project is appropriate for graduate-level study, combining
computer science with interdisciplinary applications. Since the project is
intended as a tool rather than for research purposes, a \add{video} user manual is essential
to guide users in effectively using the tool.

\newpage
\printbibliography

\end{document}
